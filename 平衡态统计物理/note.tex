\documentclass[a4paper, 10pt, openany]{book}%标注了文档类型和字号大小
\usepackage{geometry}

\geometry{left=3cm,right=3cm,top=3cm,bottom=3cm}

\usepackage{titlesec}
\usepackage{hyperref}
\usepackage{url} % 用于更好地格式化URL

\titleformat{\section}[block]{\normalfont}{\thesection}{1em}{}
\titleformat{\subsection}[block]{\normalfont}{\thesubsection}{1em}{}
\titleformat{\chapter}[display]
  {\normalfont\huge} % 设置章节标题的字体为正常字体(非加粗)
  {Chapter \thechapter}{1em}{} % 保留 "Chapter" 和 章节编号

\usepackage{xeCJK}
\usepackage{footmisc}
\usepackage[UTF8]{ctex}
\usepackage{amsmath}
\usepackage{subfigure}
\usepackage[graphicx]{realboxes}



\setCJKmainfont{FangSong}%中文设置为仿宋

\setmainfont{Palatino Linotype}%英文设置为palatino linotype

\begin{document}
  \title{ \heiti 平衡态统计物理笔记}
  \author{亦可}
  \maketitle
  \tableofcontents


  \newpage

  \chapter{写在前面/Foreword}
  \chapter{热学回顾/Thermodynamics}

  热学是唯象的,唯象的观点没有所谓对体系微观细节的认知,热力学定律基本来自实验和日常规律的总结。

  统计物理和热力学研究的核心问题,总是假定体系处于平衡状态。它们没有办法回答“如何达到平衡态”的问题。

\section{热平衡状态}
平衡状态:体系的性质随着观察的时间的推移不发生变化。

这是一个很tricky的说法。什么样的性质不随时间变化?如果我们试想一箱子气体,随着时间的推移,箱子中某一粒子的动量(或者位置)当然是随时变化的。因此,我们所谓的不随时间变化,实际上已经隐含了宏观测量的描述。测量的特征时间要比微观的运动时间长的多。我们测量的物理量一定是粗略的,缓慢的。

对于一个宏观系统,其自由度当然是很多很多的,要完整地描述它需要极多的自由度。因此在我们所谓意义上的测量时,我们是提取了体系的一个特征来进行测量,即将一个高维的相空间简化为了一些简单的热力学变量来进行测量。这些热力学变量即是符合前述“缓慢的、粗略的”定义。

有些热力学变量很直观,例如压强(描述了气体的力学特征),体积(描述了气体的几何特征)等。但这些特征并非热力学体系独有。那么,什么变量是跟“热”相关的特征量呢?我们当然已经知道这就是温度。

\subsection{第零定律}

考虑三个系统A,B与C。A与C热平衡,且B与C热平衡,则可以推出A与B热平衡。这就是第零定律。

注意,我们并没有要求ABC三个系统的相态。不过为了简单起见,我们可以考虑三个系统均为气体。

第零定律如何说明了温度的存在?

A与C热平衡,说明存在一个函数$f_{ac}$,使得$f_{ac}(p_A,V_A,p_C,V_C)=0$,同理存在$f_{bc}(p_B,V_B,p_C,V_C)=0$


\end{document}