\documentclass[a4paper, 10pt]{article}
\usepackage{geometry}
\geometry{left=3cm,right=3cm,top=3cm,bottom=3cm}
\usepackage{footmisc}
\usepackage[UTF8]{ctex}
\usepackage{subfigure}
\usepackage[graphicx]{realboxes}

\begin{document}
\title{{\huge \textbf {何述}}}
  \author{亦可}
  \maketitle

\section{魔笛}
故事。

人的自由是放弃的自由。

这是我与何述的共识。虽然我们从未对此进行过深入的讨论,但我相信他也认同这一点。

我讨厌湖边的柳树。坐在湖边的长椅上,眼前耷拉着的柳枝是仅有的看不见倒影的物。模糊的东西在势阱里作着振动,破坏了眼前世界的对称性。夜晚的湖面是无光的,如果忽略远处环湖路灯倒影的微小褶皱,这几乎是完美的。在这园子里的四年,我已经记不清有多少个晚上是在这把椅子上度过的,黑色的湖面似乎能吸收情绪的波前并反射以平静。

“你珍视自己的生命吗?”何述没头没脑地问。

“你要跳下去吗?投湖很文艺,但是就死亡的痛苦程度来说,这并不是一个轻松的方法。”我给出建议。

“没,没。只是好奇。或者换种问法,你觉得生命是值得被珍惜的吗?”

我没有给出回答。

“我给你讲个故事吧。权当临别赠礼,不成敬意。”
    
我们彻夜相谈。以至于翌日发生的两件大事,我们两个却都错过了。
    
院毕业典礼和星笛三号上天。

    
“星笛计划”是中国官方的称呼。实际上,这个计划是由中美欧三方联合推进的。说的很神秘,但是实际上,按照老板的说法,就是把19世纪的玩意捣鼓到天上去——在太阳系的尺度上造一架迈克尔逊干涉仪。听起来很简单,但这个计划从上马到如今落地也有十余年的时间了。最初,中美欧三方各自都有一个类似的计划。但是推进过程中,三方各自都发现,仅凭一方的推进,每一方都要耗费巨大的研究成本和海量的资金投入。考虑到该项目对于天文观测领域的普遍意义,最终三方各自保留项目名称,但将计划的组织架构进行了重组,并专门设立了联合工作委员会进行统筹。
    
至少这是官方的版本。
    
(真实却比较黑色幽默的事实是,其实美国在项目初期推进地非常顺利,NASA和一家私企合作,迅速攻克了许多问题,也最早证明了该计划的可行性。但是,一项关键的技术迟迟没有被突破(事后证明是方向错误)。恰在此时,中国有个留学生正在美国的K所做博后,无意间跟导师谈起之前做过的一篇研究——随后他在跟导师的交流中敏锐地意识到这可能是一种技术突破的方向。之后的短短两年不到,中国便迅速掌握了这项技术。与此同时,与NASA合作的私企不堪支撑高额的成本投入,要求与中国合作推进项目。中美的沟通率先搭建,欧洲的加入是中美妥协的结果:中国原本只打算做一个尺度较小的太空干涉仪项目,实现相对短平快的装置落地,美国的技术使得一次性落地一个大型太空干涉仪的计划成为可能,而要求就是欧洲的共同参与(毕竟共同参与意味着干涉仪的各种测量数据都能被三方共同分享,这对于未来在天文方面的学术研究话语权有着重要的奠基性意义)。)
    
老板就是天文所的。因此现下他忙的要命,而且似乎比其它同所的教授们更忙。原本毕业典礼中预备的教授代表发言也被他推掉了——这样一看,毕业典礼出席与否对许多人来说都无关痛痒。就像老板说的:“毕业典礼是一个纪念性的仪式。我接受仪式感作为一种价值判断的存在性,但我确实不能感同身受。我坚定地认为纪念、回忆或者缅怀等类似的情绪是没有意义的。”

这一点上,我很赞成老板。我是个沉默的人,很多时候,我不愿意将自己的观点跟别人表述出来。但是老板不一样,我从来不曾见到他隐藏任何想法。当然这有时候表现地很尖刻(倒不如说是尖锐,因为显然他并没有试图让你不舒服,他只是指出事实),但是你知道这是实话,也就不觉得刺耳了。等你回头细想,你会觉得解脱。

“我讨厌任何虚与委蛇。当你试图对一个人表述一句话的时候,你应该确保这句话中的全部信息被对方理解。你应该为信息传播的完整度负责。在此基础上,我喜欢简洁。”是这样的,我从来未曾不理解过老板的任何一句话,如果有,那最终会被证明是我想多了。这是我为什么当初听了两次组会就跟了他的原因,必须承认,这种对话风格极富个人魅力。

但我还有必须承认的事。老板有的时候真的过于尖锐了,以至于他“总是”会让你面对你试图逃避的问题。在关键问题上,这很重要也很有用;但是当你只是试图图个轻松的时候,这种尖锐往往让人扫兴。因此,我只能跟老板成为“一点点”的朋友。这是老板盖棺定论的。

“我清楚没有人能跟我成为完全的朋友。但是即使是一点点的朋友我也十分荣幸。”

我这样想着,坐在了吧台附近的角落里。是的,如果直到下午才醒,那我的后半天就会选择来这家轻吧坐坐。装潢没有一般吧台的斑斓和妖艳,低衬比度的暖光被泼在墙面上,也泼在那面放酒的架子上,照出一排排黑漆漆的藏酒,简单、沉默、温暖。店大客少,这种周中的晚上营生都比较惨淡,只有零星的常客才会来这里。常客来这里已成了习惯,既不是为了一定要喝酒,也不是为了一定要点歌,所以驻唱在周中一般也不来浪费时间。总不能太冷清,所以店里用音响放着音乐。正如店主说的,音乐就像酒,有的人喜欢烈酒,有的人喜欢扎啤。所以,卖酒的和卖唱的本质上是一种人:摸清市场需求,然后制作产品。只能说,这家店成功了。我既爱听它们放的歌,也爱喝它们卖的酒。同时很幸运地,四年来,酒保成了我完全的朋友。

“马天尼?”酒保把一杯尖果递到我面前。

“不了,椰林飘香吧。”我顺手拿起一颗放进嘴里,咯吱咯吱的声音顺着骨头传进耳膜。

“怎么改旧口味了。喝了好几个月的金酒喝腻了?”

“没有。感觉你最近调的有点酸了,不太喜欢。”

“OK。你多等会,今天人比较少,我也弄一杯咱俩一块喝。”

“我不着急啊,反正最近事一直比较少。”我把笔记本放在桌上打开,随意地浏览着新闻。

“事少吗?我都看见新闻了,你们这种内部人士还能不知道?那魔笛三号上天,你们不得有的忙了,是不是?我记着你之前跟我说,这仨卫星能搞什么天文观测来着,这事能少得了你们?”酒保已经回吧台开始shake了,还是什么,我也不懂。

“是我老板有的忙了,我都毕业了还忙啥。”

“诶你毕业啦,也是,咱俩认识也得有四年了。我还以为你要继续在这儿读呢?所以你后面不跟他干了?”

“不知道啊。我现在想着是先gap一年,真是有时候活着啥都不想干,就想每天下午起凌晨睡,在你这酒馆蹲一晚上。然后后半夜你给我捡回你家睡一晚上,下午醒了继续过来喝。”

“幽默了兄弟,我倒是希望全北京的人都像你这么有觉悟呢,那样我的收入就有保障了。”

“好啊,到时候分我一半觉悟费就行。我好赞助我老板点,让他最近出什么新文章的时候给我挂个名,我估计他们最近能发几篇大的。”

“包的呀,不过说到你老板要出文章,就得跟今天这卫星有关系了吧,是不是?最近这个魔笛三号,我看网上吹的神乎其神的,这玩意怎么个牛逼法到现在我也没被科普明白。”

我沉默。我其实对这三个卫星的内部结构和技术细节也不甚了解,不过大致上,我明白它们工作的原理。“我打个比方吧。你觉得你能听到这个世界上所有的声音吗?或者我换个问法,你觉得是什么阻止了你听到更多的声音?”

“这不是很简单吗?这儿有堵墙,隔壁说话我就听不见了。”


“还有呢?”

“要是驻唱在这儿,然后人多点儿,有时候打碎了一两个杯子也听不见,得等夜场结束了人都溜了才能看见地上的玻璃碴子。”

“还有吗?”

“这当然能举很多例子,这有什么用吗?你不得给我解释解释?”

“OK,其实你说的这些东西基本已经概括的比较全了。我们现在来做一下理论的抽象。一堵墙放在这里,为什么你就听不见了?是不是任何障碍物都有这种效果?这种“听不见”是因为声音被屏蔽了,你站在一个屏蔽空间里,所以接收不到声音。”

“有点明白了,你接着说。”

“听不见打碎杯子的声音,是因为噪声太多了,这是你的耳朵对复杂声音的分辨能力不够,对不对?”

“这确实是。还有别的吗?”

“你忽略了一个相对来说不是很常见的现象。我们是听不见海豚的声音的。人耳的接收有个频率范围,超出这个范围的声音你也感受不到。”

“明白了。也就是说,要是有个人在我面前弹钢琴,要是我的耳朵的接收频率范围小到一定程度,那我就只能听见他弹do这个音,他摁其他键我一点也感觉不到,是不?”

“完全正确的。魔笛相比于我们早先的引力波探测仪,主要就是强在了第二点和第三点。首先你得清楚,你要是想听摔杯子的声音,你最好等到安静的时候再摔。魔笛也一样,地球上的噪声太多了。我们要是想听清宇宙的声音,就应当把耳朵直接放在宇宙的尺度上。”

“其次呢?”

“其次就是关于这个仪器本身的。这个仪器有两个长臂,这俩东西越长,它的量程就越大,灵敏度就越高。我们把这玩意建在太空里,它的臂长就是AU尺度的,相当于能听见原来听不见的频率,还听得更清楚。”

“明白了。所以我们是弄了一个大眼睛上去看星星。”

“对的。诶,酒好了没有,说这么多嘴有点儿干。”

“OK你的好了,拿着。”

我猛吸一口。椰子的那股清香和酒的微微的刺激感进到舌根,冰凉的感觉一路向下。突然,页面上显示我收到了一封邮件。我点开,是老板发的。很简单,四个大字配一行小字,但是很不老板,因为我在彼时彼刻没能完全理解这句话的全部意思。

“\textbf{拨云见日。}另:有事详谈,明晚8至9点请到北301。若无时间请在此前回复。——舒”

\section{拨云见日}
还差十分钟八点的时候,我敲响了老板的办公室。

“请进。”

我推门。凌乱的办公室一如往常,唯一与往常不同的是工位旁多了一个灰色的行李箱。

“您要走了?”这是我能得出的唯一结论。

“在走之前,我有事情要交代给你。这就是今天叫你来的原因。请坐吧。”他随手指了指角落里的那把椅子,我费力地把它从一堆杂物中拽出来。

“原因和走的时间?”

“明天晚上的高铁,具体去哪不便透露。原因我会告诉你。请给我几分钟陈述时间。”

“简单地概括我们面临的问题:我们发现了一组信号。我们认为这是地外文明的痕迹。魔笛吹响第一小节时,我们还不太确定;但是到现在,我基本可以确定,这一定是某种“非自然”的信号。于是问题接踵而至。”

\vspace{10pt}

/一些问题

\vspace{10pt}

“我们长话短说。魔笛奏出了某种音乐。虽然现在我们还不能理解,但看过这个波形图的人都相信这是某种信号,至少我们不能说服自己这是完全的噪声。”

老板把笔记本转到我面前。这是一列脉冲波。老板的描述很准确,即使是最拒绝承认这是一串信号的人,也不能说服自己承认这就是毫无意义的噪声。

“所以您的意思,我们听到了外星人的语言。”

“这改变了一切。我们在事前完全没有这方面的准备,因此现在一切都显得很仓促。你这两天见过何述了吗?”

“前天晚上我和他在一块。”

“太早了,现在瞬息万变。他也要离开北京了,我也要离开北京了。”

“去哪?”

“我和何述的工作都不便与你透露,但我有事需要你帮我去做。我会给你实时接入魔笛的权限,”

\vspace{10pt}

/一些问题

\vspace{10pt}

在花店门口,我第一次见到了舒离。

老实说,我有些脸盲,但是第一次遇见舒离我就知道我不会认错她。比脸更首先被人发觉的是扑面的、极自然的花香。老实说我很喜欢这种香气,它让我想起老家村口槐花开时的香气(虽然其实我已经不大记得具体的感官刺激了,而且实际上村里的土路上一年四季往往都弥散着淡淡的农家肥的味道)。她对我点了点头。

“呃,请问您是陆然哥哥吗?”

“是我。您叫我小陆就行,我得管您哥叫老板呢。”

“别介,我哥是我哥我是我,而且这不还得拜托您帮忙搬东西吗。这的东西不多,主要是到时候得从家里往过搬。您稍等,我进去把这趟的东西拿出来。”

“好的,需要我帮你吗?”

“不用,就一点东西。”

于是我就在店外等着,顺便打量着这家店铺。

不得不说,花店的装修很精致,布局也很考究。唯一的问题是店面太小了,看得出来舒离在设计上尽量让它显得敞亮一些,但毕竟要展示一些花艺,狭小的空间还是略微让人喘不过气。

至于展品,舒离选择缩小展示的尺寸。整个店里除了几个微缩的模型,没有用包装纸作扮的精美的花簇成品,基本都是在展示格里的一小杯一小杯的样式。看得出来,舒离似乎不是很喜欢红色。整个店里除了两三株水培的蝴蝶兰有点极浅的粉色,整个店里没什么红色系的花。

舒离从店内出来,提着一个行李箱,箱子上搭着两个纸袋子。舒离把两个袋子递给了我:“这个,店里养的玉坠,我哥一份你一份吧。放心吧,这两盆都不是药锦,随便养养就能活。”

“药锦是啥意思?”

“就是用些药洒在这种多肉上,能抑制叶绿素的分泌还是啥,让多肉显得透亮一点,但很损寿命的,几个月就活不了了。”

“还能这样。那买药锦的意义在哪里呢?”

“很多人买花又不是真的想种。至少在我店里卖的药锦我都会注明。很多人来我这买药锦就想买一杯奶茶一样,带回去赏几天就丢掉,连水也不用浇,图个暂时的欣赏罢了。”

“我还是第一次知道有人这么买花,感觉有点快餐文化了。”

“听起来有点贬义哦,看来你不是很赞成这种做法啊。”

“难道不是吗?我们正在逐渐失去对生活的耐心啊,宁可花钱买一株注定马上就要死掉的药锦,也不愿意花时间好好养一盆健康的花,这还不够感慨吗?”

“陆哥,无病呻吟莫过于此吧。

“何以见得?”

“我给你讲个故事吧。

两年前的时候,我的店还没有这么精致,设计也没有这么好,虽然我当时和现在一样用心,但不同之处在于当时的我在卖花的时候总是真心希望买花的人回去会和我一样把花养好。所以,当时我也不会卖那些用精美包装纸做好的花团,因为很显然,少有人愿意把这种礼物性的花拿回去养着,哪怕只是插个水瓶都不愿意。

转变发生在那年除夕我给房东奶奶送花那天,我跟她抱怨现在花店生意的惨淡。她带着那种洞悉的眼神,跟我说,孩儿啊,不是每个人都愿意花大把精力在养花上的,你应该给顾客提供符合他们需求的选择。我反驳她,我就是想让大家的生活慢下来。

她说,那你就别再抱怨了。人们是不愿意听说教的。而当你在对别人做他们不愿意让你做的事情时,你为此支付一些成本也是合理的。

她说,你应该想想,你开店到底是为了宣传一种慢节奏的生活方式,还是真的安安心心为人们提供一些简单的快乐?如果是前者,那你就把亏损认为是宣发的费用;如果你真的愿意售卖些简单的快乐,那就真的给客人提供些合适的选择。

房东奶奶是爱花的人。我大大小小送了她十几坛花,她都养的极好,此外自己还侍弄着些古色古香的盆栽,所以这番话从她口中说出来,我知道我也许真的应该做些改变。

所以,我真的开始认真学习除了花艺以外的东西,房屋设计、书法、还有一些简单的美术原理,把店打扮成了现在这个样子,也给客人提供了更多的选择。

我的观点很简单。显然我不比消费者更懂他们想要什么,所以我尽量给他们提供更多的选择,至于你说的什么感慨,我不能苟同。”

“我只是觉得,有些东西不该被我们丢掉。”

“你错了,所有由于你觉得不该被丢掉而被你坚守着的东西,都只是你自以为坚守着罢了。过去没有任何意义。”

\vspace{10pt}

/一点问题

\vspace{10pt}

“所以,你知道我哥之后要去干什么吗?”

“不知道,老板没跟我说,我以为他会告诉你。”


\end{document}